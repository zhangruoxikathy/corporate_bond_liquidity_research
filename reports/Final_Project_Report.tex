\documentclass{article}
\usepackage[english]{babel}
\usepackage[letterpaper,top=2cm,bottom=2cm,left=3cm,right=3cm,marginparwidth=1.75cm]{geometry}
\usepackage{amsmath}
\usepackage{booktabs}
\usepackage{geometry} % For setting margins
\usepackage{graphicx}
\usepackage{longtable}
\usepackage{float}
\usepackage{multirow}
\usepackage{setspace}
\usepackage[colorlinks=true, allcolors=blue]{hyperref}
\geometry{a4paper, margin=1in}

\newcommand*{\PathToAssets}{../assets}%
\newcommand*{\PathToOutput}{../output/}%


\title{Replication of The Illiquidity of Corporate Bonds: Project Overview and Table Results}
\author{Group 10: Arthur Ji, Hantao Xiao, Hunter Young, Kathy Zhang}

\begin{document}
\maketitle

\begin{abstract}
Our project set out to replicate Tables 1 (Summary Statistics) and 2 (Measure of Illiquidity) from "The Illiquidity of Corporate Bonds" by Bao, Pan, and Wang (2010). This seminal paper evaluates the impact of illiquidity on corporate bond pricing, employing a novel measure of illiquidity, $\gamma$, for each bond. Focusing on corporate bonds from 2003 to 2009, the study meticulously calculates illiquidity measures and analyzes their valuation effects.
\end{abstract}

\section{Overview}

In the paper, Table 1 generates summary statistics for all corporate bonds and selected samples during 2003 - 2009, and Table 2 calculates illiquidity measure $\gamma$ at both individual bond level and portfolio level. 

In addition to replicating the original tables, we introduced our own supplementary statistics and visualizations of calculated bond illiquidity to further elucidate the data. These enhancements aim to provide a more comprehensive view of the datasets and their implications for corporate bond illiquidity. 

\subsection{Data}

In order to replicate and automate both tables, we leverage four data sources:

\begin{enumerate}
  \item \textbf{WRDS BondRet dataset:} A cleaned database incorporating two feeds: FINRA’s TRACE (Trade Reporting and Compliance Engine) data for bond transactions, and Mergent FISD data for bond issue and issuer characteristics, reported on a monthly basis.
  
  \item \textbf{Daily TRACE panel data:} Maintained by a group of contributors from \href{https://openbondassetpricing.com/}{Open Source Bond Asset Pricing}, this data includes individual level price-relevant data based on FINRA’s TRACE data, reported on a daily basis.
  
  \item \textbf{FINRA’s TRACE data:} The original raw data containing individual level bond characteristics, reported on a trade-by-trade basis.
  
  \item \textbf{MMN-corrected WRDS TRACE data:} The bond-level panel with characteristics adjusted for market microstructure noise, pulled directly from Open Source Bond Asset Pricing, reported on a monthly basis.
\end{enumerate}

\subsection{Replication Results}
Table 1 was reconstructed using data from WRDS BondRet and the original TRACE, including all necessary summary statistics except for trade numbers and sizes, derived from the latter. For Table 2, the daily illiquidity measure leveraged the Daily TRACE panel and MMN-corrected panel, while trade-by-trade illiquidity and bid-ask spreads utilized data from the original TRACE and WRDS BondRet, respectively.

We are successful in replicating the whole process of generating the two tables, applying the filters of sample selection outlined in the paper, and generating similar results compared to the original paper. As informed by our unit tests, our results in the two tables are close to the original paper in terms of absolute values, or, at least, data trends. Additionally, we incorporated the latest data to refresh the tables, capturing recent market dynamics.

\subsection{Challenges}
However, challenges arose due to the limitations of the original datasets. The 2010 paper relied exclusively on TRACE data, which later research suggested might introduce bias due to short-term price reversals. Also, processing the extensive dataset of 346 million trades from 2003 to 2009 was time-intensive. To mitigate these issues, we primarily used pre-processed data from WRDS BondRet and the Daily TRACE panel, which have addressed these reversal effects. This approach, while necessary, occasionally resulted in discrepancies from the original figures due to the different data sources and the exclusion of some transactions recorded in the original TRACE data. We also employed MMN-corrected WRDS TRACE monthly bond data to reconstruct the Table 2 Panel A daily data table, which was a crucial update mentioned on open source bond asset pricing website to adjust for market microstructure noise. After MMN correction, the illiquidity measures are overall lower with higher standard deviation over years.

Updating the results to the current period revealed that the methodology's exclusion of post-Phase 3 bonds (after February 7, 2005) significantly reduced the dataset over time, and certain bond filtering indicates bonds used in 2003-2009 may lose its ability to be included for the updated table, casting doubt on the recent relevance of the illiquidity measures. 


\section{Tables}

\subsection{Table 1 Summary Statistics}

Table 1 provides a detailed overview of the study's sample, comprising frequently traded Phase I and II bonds from April 2003 through June 2009. As detailed in Panel A, the sample includes approximately 800 bonds annually within the specified period, though the total number fluctuates year to year. The observed increase in bond numbers from 2003 to 2004 and 2005 is likely due to NASD's expanded coverage to include Phase III bonds, whereas the decline from 2004 to 2009 can be attributed to bonds maturing or being retired. This fluctuation mirrors the trends observed in the original data.

The bonds featured in the sample are substantial, boasting a median issuance size of around \$750 million, and are predominantly investment grade, with a median Moody’s numeric rating between 5 and 6 throughout the years. In contrast, Panel B, covering all bonds in TRACE, presents a lower median issuance size and rating, as anticipated.

With an average time to maturity of nearly 6 years and an average age of about 4 years, the sample shows a gradual decrease in maturity and an increase in age over time, a consequence of the sample selection criteria excluding bonds issued after February 7, 2005, marking the onset of Phase III.

The criteria for selecting the bonds suggests they are traded more frequently than average. Notably, in 2008-2009, Panel B's average turnover ratio for a bond was higher, although the median was considerably lower, indicating outliers' influence on the mean. In terms of the number of trades, average trade size, and turnover ratio, the bonds in Panel A demonstrate slightly higher figures compared to Panel B, indicating enhanced liquidity.

The average return of the bonds, according to our calculation, is lower than that reported in the original paper. However, the trend of average returns from 2003 to 2009 closely aligns with the original paper, showing a significant drop during the Global Financial Crisis in 2008, followed by a recovery in 2009. The volatility and price of the bonds in both Panel A and B closely resemble those in the original paper.


\begin{figure}[hbt!]
\centering
\textbf{\large Table 1 from the paper}
\includegraphics[width=0.75\textwidth]{\PathToAssets/table1_screenshot.jpg}
\end{figure}


\subsubsection{Replicate Tables 1 in the Paper, For period 2003/04-2009/06}
\doublespacing


\begin{table}[hbt!]
\centering
\textbf{\large Panel A: Bonds in Our Sample, 2003-2009}
\resizebox{\textwidth}{!}{%
    \input{\PathToOutput table1_panelA.tex}
}
\label{table:table1_panelA}
\end{table}


\begin{table}[hbt!]
\centering
\textbf{\large Panel B: All Bonds Reported in TRACE, 2003-2009}
\resizebox{\textwidth}{!}{%
    \input{\PathToOutput table1_panelB.tex}
}
\label{table:table1_panelB}
\end{table}


\subsubsection{Update Table 1 in the Paper, For period 2009/06-Present}
\doublespacing

\begin{table}[hbt!]
\centering
\textbf{\large Panel A: Bonds in Our Sample, 2003-Present}
\resizebox{\textwidth}{!}{%
    \input{\PathToOutput table1_panelA_uptodate.tex}
}
\label{table:  table1_panelA_uptodate}
\end{table}


\begin{table}[hbt!]
\centering
\textbf{\large Panel B: All Bonds Reported in TRACE, 2003-Present}
\resizebox{\textwidth}{!}{%
    \input{\PathToOutput table1_panelB_uptodate.tex}
}
\label{table:  table1_panelB_uptodate}
\end{table}



\subsection{Table 2 Measure of Illiquidity $ \gamma = -\text{Cov}(p_t - p_{t-1}, p_{t+1} - p_t) $ }


\begin{figure}[hbt!]
\centering
\textbf{\large Table 2 in the paper}
\includegraphics[width=0.75\textwidth]{\PathToAssets/table2_screenshot.jpg}
\end{figure}


\subsubsection{Replicate Table 2 in the Paper, For period 2003/04-2009/06}
\doublespacing
\begin{table}[hbt!]
\centering
\textbf{\large Panel A: Individual Bonds, Trade-by-Trade Data, 2003-2009}
\input{\PathToOutput table2_panelA_trade_by_trade_paper.tex}
\label{table:table2_panelA_trade_by_trade_paper}
\end{table}

\begin{table}[hbt!]
\centering
\textbf{\large Panel A: Individual Bonds, Daily Data, 2003-2009}
\input{\PathToOutput table2_panelA_daily_paper.tex}
\label{table:table2_panelA_daily_paper}
\end{table}

\begin{table}[hbt!]
\centering
\textbf{\large Panel B: Bond Portfolios, 2003-2009}
\input{\PathToOutput table2_panelB_paper.tex}
\label{table:table2_panelB_paper}
\end{table}

\begin{table}[hbt!]
\centering
\textbf{\large Panel C: Implied by Quoted Bid-Ask Spreads, 2003-2009}
\input{\PathToOutput table2_panelC_paper.tex}
\label{table:table2_panelC_paper}
\end{table}


\subsubsection{Update Table 2 in the Paper, For period 2003/04-Present}

\begin{table}[hbt!]
\centering
\textbf{\large Panel A: Individual Bonds, Trade-by-Trade Datas, 2003-Present}
\resizebox{\textwidth}{!}{%
    \input{\PathToOutput table2_panelA_trade_by_trade_new.tex}
}
\label{table:table2_panelA_trade_by_trade_new}
\end{table}

\begin{table}[hbt!]
\centering
\textbf{\large Panel A: Individual Bonds, Daily Data, 2003-Present}
\resizebox{\textwidth}{!}{%
    \input{\PathToOutput table2_panelA_daily_new.tex}
}
\label{table:table2_panelA_daily_new}
\end{table}


\begin{table}[hbt!]
\centering
\textbf{\large Panel B: Bond Portfolios, 2003-Present}
\resizebox{\textwidth}{!}{%
    \input{\PathToOutput table2_panelB_new.tex}
}
\label{table:table2_panelB_new}
\end{table}


\begin{table}[hbt!]
\centering
\textbf{\large Panel C: Implied by Quoted Bid-Ask Spreads, 2003-Present}
\resizebox{\textwidth}{!}{%
    \input{\PathToOutput table2_panelC_new.tex}
}
\label{table:table2_panelC_new}
\end{table}


\subsubsection{Table 2 Panel A Daily Data using MMN-Corrected Bond Data}


\begin{table}[hbt!]
\centering
\textbf{\large Panel A: Individual Bonds, MMN-Corrected Bond Data, 2003-2009}
\input{\PathToOutput table2_panelA_daily_mmn_paper.tex}
\label{table:table2_panelA_daily_mmn_paper}
\end{table}


\begin{table}[hbt!]
\centering
\textbf{\large Panel A: Individual Bonds, MMN-Corrected Bond Data, 2003-Present}
\resizebox{\textwidth}{!}{%
    \input{\PathToOutput table2_panelA_daily_mmn_new.tex}
}
\label{table:table2_panelA_daily_mmn_new}
\end{table}


\subsection{Monthly Bond Illiquidity Summary Statistics}

By integrating the summary statistics and graphical representations of monthly bond illiquidity, we gain insight into the pronounced fluctuations in illiquidity, particularly during the 2008 sub-prime crisis. The summary statistics offer a comprehensive view of the annual distribution of illiquidity, highlighting extreme peaks of over 5000 in 2008 and over 8000 in 2009. In contrast, the third quartile remained between 1 and 2, indicating the presence of significant outliers, likely driven by a few bonds with gamma values surpassing 2000. Despite these spikes, the median values of 0.23 and 0.33 in 2008 and 2009, respectively, were considerably lower than the mean values of 13.27 and 17.98, suggesting a generally higher liquidity level during these years. Extending the analysis to 2023 revealed notable surges in illiquidity in 2018 and 2020.

The scatter plots visualize monthly illiquidity for individual bonds, with the red and purple lines delineating the mean and median annual illiquidity, respectively. These plots uncover a declining trend in monthly bond illiquidity, succeeded by a rising trend leading up to 2009, marked by exceptionally high illiquidity instances reaching up to 8000. Closer examination of the subsequent plot, which zooms in on the data, shows most values clustering in the 0-200 range. Notably, the mean illiquidity consistently exceeds the median during 2007-2009, reflecting a positive skew in the data attributed to outliers. The period post-2010 is characterized by more stable fluctuations, possibly due to a decrease in bond numbers following phase III. The MMN-corrected dataset presents a more uniform distribution, reducing outlier impact while preserving the identified trends.


\begin{table}[hbt!]
\centering
\textbf{\large Monthly Bond Illiquidity Summary Statistics Using Daily Data, 2003-2009}
\input{\PathToOutput illiq_summary_paper.tex}
\label{table:illiq_summary_paper}
\end{table}


\begin{table}[hbt!]
\centering
\textbf{\large Monthly Bond Illiquidity Summary Statistics Using Daily Data, 2003-Present}
\resizebox{\textwidth}{!}{%
    \input{\PathToOutput  illiq_summary_new.tex}
}
\label{table: illiq_summary_new}
\end{table}


\begin{table}[hbt!]
\centering
\textbf{\large Monthly Bond Illiquidity Summary Statistics Using MMN-Corrected Bond Data, 2003-2009}
\input{\PathToOutput illiq_daily_summary_mmn_paper.tex}
\label{table:illiq_daily_summary_mmn_paper}
\end{table}


\begin{table}[hbt!]
\centering
\textbf{\large Monthly Bond Illiquidity Summary Statistics Using MMN-Corrected Bond Data, 2003-Present}
\resizebox{\textwidth}{!}{%
    \input{\PathToOutput illiq_daily_summary_mmn_new.tex}
}
\label{table: illiq_daily_summary_mmn_new}
\end{table}


\section{Visualizations}

\subsection{Monthly Illiquidity Per Bond and Average Illiquidity By Year (See combined observations in the previous section)}


\begin{figure}[hbt!]
\centering
\caption{Illiquidity by Year with Mean Illiquidity, 2003-2009}
  \centering
  \includegraphics[width=0.75\linewidth]{\PathToOutput/illiq_plot_2003-2009.png}
  % \caption{Average returns}

\label{fig:illiq_plot_2003-2009}
\end{figure}


\begin{figure}[hbt!]
\centering
\caption{Illiquidity by Year with Mean Illiquidity, 2003-Present}
  \centering
  \includegraphics[width=0.75\linewidth]{\PathToOutput/illiq_plot_2003-2023.png}
  % \caption{Average returns}

\label{fig:illiq_plot_2003-2023}
\end{figure}

\end{document}